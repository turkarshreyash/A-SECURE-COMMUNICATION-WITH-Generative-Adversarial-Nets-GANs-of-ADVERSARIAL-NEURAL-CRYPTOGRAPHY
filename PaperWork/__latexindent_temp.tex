
\documentclass[journal,transmag]{IEEEtran}

\usepackage{amsmath}
\usepackage{upquote}
%\usepackage{adjustbox}
%\usepackage{showframe}
\usepackage{graphicx,wrapfig,lipsum,algorithm,algorithmic,pgfplots,booktabs}

\hyphenation{op-tical net-works semi-conduc-tor}
\usepackage{amsmath}
\usepackage{upquote}
\usepackage{booktabs}
\usepackage{multirow}

\usepackage{graphicx,wrapfig,lipsum,algorithm,algorithmic,booktabs}

\begin{document}
	
	\title{A SECURE COMMUNICATION WITH Generative Adversarial Nets (GANs) of ADVERSARIAL NEURAL CRYPTOGRAPHY}
	
	
	\author{\IEEEauthorblockN{Shreyash Turkar\IEEEauthorrefmark{1}, Nidhi Singh\IEEEauthorrefmark{2},
		}
	
			\IEEEauthorblockA{\IEEEauthorrefmark{1,2}Department of Computer science \& Engineering,\\
		 Indian Institute of Information Technology, Nagpur Maharashtra, 440003, India}
		
		shryeash.turkar@iiitn.cse.in, nidhi.2592@gmail.com
		
	}
	
	
	


	

	\maketitle.

	\begin{abstract}
		Abstract: \input{./TextContent/Abstract.txt}
	\end{abstract}
    
    % Note that keywords are not normally used for peer review papers.
    \begin{IEEEkeywords}
    	Symmetric cryptography, GANs, CNN
    \end{IEEEkeywords}
    
    
    
    % make the title area
    %        \maketitle
	\section{Introduction}
	\input{./TextContent/Introduction.txt}
    \par
     Our contribution in this paper can be summarized as follows:


	 \section{Symmetric Encryption}
	     
    
    
    
    
    
    
    
    
    
    
    \section{Related Work}
    Smart cities are one of the tremendous research area under IoT in today’s world \cite{Gupta}. 
    
    \section{Proposed Work}
    The proposed work consists of two sub-parts. In the first section, we discuss the real-time images capturing task with the installed cameras. However, in the second section, we demonstrate the continuous display device of road conditions by handover mechanism among the APs. The SRM methodology collaborates with the IEEE 802.11 (Wi-Fi module) to provide the connectivity between a vehicle and APs. It can be seen as equation \ref{q1}
    \begin{equation}
    a^2+b^2
    \label{q1}
    \end{equation}
    
    \subsection {Real-time capturing of road conditions}
    
    Nowadays, providing safety on the roads is one of the major concern in smart cities. The term safety can be used to indicate the traffic issues and road conditions in real-time. The users are more concerned with the traffic analysis before moving to a particular direction of the road. In addition, the security surveillance system focus on continuous eye-point on the road for un-ethical events. Therefore, we proposed SRM strategy for real-time analysis of traffic situations and road conditions.
    \section{Performance analysis}
    In this section, a performance analysis is made for proposed SRM approach with a comparison of V2V and LTE-V2V technologies scenario. We used Netsim simulator for the implementation of the modules of the proposed working scheme. In the simulation setup, we used 10 cameras in which each range is 30 meters. In addition, the AP is located each after the 2$^{nd}$ camera. We used Wi-Fi technology to provide the connectivity between AP and ongoing vehicle. In the event of changing the AP, an efficient handoff decision is made. We set up the range of each AP is 500 meters. It means, one AP can provide the real-time road condition monitoring under the area of 500 meters. For the simulation of LTE-V2V, the range of LTE is set to be 400 km. The velocity of the ongoing vehicle can vary between 10-100 m/s. The size of the packet is set to be 130 bytes. We used random walk mobility model with the utilization of the feature of range propagation loss model. At last, the simulation time is set to be 800 seconds.
    
    \subsection{Simulation results in terms of collision probability}
    
    In Figure \ref{p1}, the simulation result in terms of collision probability is shown. This performance metric is directly linked with the safety of drivers on the roads. If the probability is less, then the scheme is said to be effective in the provision of providing the safety on roads. In the figure, the graphical analysis of collision probability is shown with respect to the distance between the source of starting vehicle towards the destination. From the figure, it can be seen that the proposed scheme SRM outperforms by achieving the less probability of collision on the road. The anticipated scheme achieves 0.30 value of probability by covering the maximum distance mentioned in the simulation setup i.e. 500 meters. However, other schemes named as V2V and LTE-V2V achieves 0.87 and 0.5 respectively by covering the same distance of 500 meters. In addition, when the total distance covered by the vehicle is 100 meters, the scheme LTE-V2V achieves approximate similar results in comparison to proposed SRM scheme by achieving 0.15 and 0.13 respectively. In addition, other schemes named as ILTs and iTAS achieves 0.52 and 0.5 respectively by covering the same distance of 500 meters. Therefore, it can be stated that the proposed approach outperforms in terms of probability of collision in compared with other scheme anticipated for the safety on the roads.
    
    \begin{figure}
    	\begin{center}
    		\includegraphics[width=8.04cm]{s3.png}
    	\end{center}
    	\begin{center}
    		\caption{Performance analysis in terms of average handoff delay}

    		\label{p1}
    	\end{center}
    \end{figure}
    	
    
    
    \section {Conclusion}
    In the current world's environment, there is a tremendous growth in technology for making the life based on IoT. However, in the secession of traffic monitoring of roads in smart cities, there is a requirement to provide a strategy that gives real-time analysis of road conditions. To cope-up with this challenge, we proposed an efficient SRM strategy for providing safety on roads by capturing of images. The cameras are located on the side of the road in order to capture the images and forwards towards the AP. The ongoing vehicles on the road are connected with these AP and easily access the condition of roads/traffic in the LED display. In this regard, we concluded that better safety can be provided by using proposed SRM strategy. The reason is, it utilizes the real-time capturing of images for making an efficient decision rather than audio signals and text messages. The simulation results showed that proposed strategy can achieve better performance in comparison to other scheme in the provision of various performance matrices. As we stated that the proposed work is simulated using the scenario of straight road condition. Therefore in our future work, we will be focused on the multi-directional road scenario such that a better safety scheme can be provided to users on the roads.
	
  
  
  
\begin{thebibliography}{00}
	
	%% \bibitem must have the following form:
	%%   \bibitem{key}...
	%%
	
	% \bibitem{}
	%\begin{enumerate}
	
	
	\bibitem{Bellavista}
	%1
Bellavista, Paolo, Federico Caselli, Antonio Corradi, and Luca Foschini. "Cooperative Vehicular Traffic Monitoring in Realistic Low Penetration Scenarios: The COLOMBO Experience." Sensors 18, no. 3 (2018): 822.
	
	
	\bibitem{Gupta}
	%2
	Gupta, Maanak, and Ravi Sandhu. "Authorization Framework for Secure Cloud Assisted Connected Cars and Vehicular Internet of Things." In Proceedings of the 23nd ACM on Symposium on Access Control Models and Technologies, pp. 193-204. ACM, 2018.
	
\end{thebibliography}


	\end{document}
	% End of v2-acmsmall-sample.tex (March 2012) - Gerry Murray, ACM
	
